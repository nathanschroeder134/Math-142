\documentclass[12pt,letterpaper]{hmcpset}
\usepackage[margin=1in]{geometry}
\usepackage{graphicx}
\usepackage{amsthm}
\usepackage{enumitem}

\input{macros.tex}

% info for header block in upper right hand corner
\name{Nathan Schroeder}
\class{Math142 FA19}
\assignment{Homework 2}
\duedate{Due: Mon, Sep  21, 2019}

\renewcommand{\labelenumi}{{(\alph{enumi})}}


\begin{document}

\begin{problem}[Problem 2 Section 1-4  - Do Carmo]
A plane $P$ contained in $\mathbb{R}^3$ is given by the equation

\[a x + by + c z + d = 0\]\

Show that the vector $v = (a,b,c)$ is perpendicular to the plane and that  $\frac{\vert d \vert}{\sqrt{a^2+b^2+c^2}}$ measures the distance from the plane to the origin $(0,0,0)$.
\end{problem}

\begin{solution}\

Technically, what this problem asserts can only be true when $d = 0$.   To say that the vector $v$ is orthogonal to the plane $P$ is to assert that 

\[ \text{for all} \  (x,y,z) \in P \ \text{ we have} \ <(x,y,z), (a,b,c)> = 0\]\

But then

\[ 0 = ax+by+cz +d = <(x,y,z),(a,b,c)> + d = d \]\

The set 

\[ P = \{(x,y,z) \in \mathbb{R}^3: ax+by+cz + d =0 \} \]\

is an example of an affine subspace of $\mathbb{R}^3$ (also called a flat).  Essentially an affine subspace is like a subspace but is not required to contain the origin.  They are closed under affine combinations of their elements, which is a linear combination with the additional assumption that the  coefficients of the combination must add up to 1.  The following fact essentially characterizes affine subspaces in terms of linear subspaces of $\mathbb{R}^3$:\\

{\large Fact:} Given an affine subspace $P$ there exists a unique subspace and a unique vector $T$ such that 
\[ P = V+T = \{ v + T : v \in V\}\]

and the distance to $P$ from the origin: 

\[d(\vec0, P) = inf \{ \Vert p \Vert_2 : p \in P \}= \Vert T \Vert_2\]\

 is given by the magnitude of the vector $T$.

\newpage

The correct statement of the orthogonality condition asserted in this problem is:

\[P \ \text{is a translation of the orthogonal complement of} \ v \]\

  Writing  
\[v^{\bot} = \{ (x,y,z) \in \mathbb{R}^3: ax+by+cz =0 \}\]\

for the orthogonal complement of $v$,  we need to find a vector $T \in \mathbb{R}^3$ such that

\[ P = v^{\bot} + T \]\

Let

\[ T = (\frac{-d}{a^2+b^2+c^2} a, \frac{-d}{a^2+b^2+c^2} b, \frac{-d}{a^2+b^2+c^2} c)\]\\


{\large Claim 1: $\quad\quad P = v^{\bot}+T$}\\

First suppose that

\[(\tilde{x},\tilde{y},\tilde{z}) \in v^{\bot} + T\]\

then

\[   (\tilde{x},\tilde{y},\tilde{z}) = (x+ \left(\frac{-d}{a^2+b^2+c^2}\right) a, y+ \left(\frac{-d}{a^2+b^2+c^2} \right) b, z + \left(\frac{-d}{a^2+b^2+c^2}\right) c)\]\

for some $(x,y,z) \in v^{\bot}$.\\

so we have 

\begin{align}
a&\left(x+ \left(\frac{-d}{a^2+b^2+c^2}\right) a\right) + b\left(y+ \left(\frac{-d}{a^2+b^2+c^2}\right) b\right) + c\left(z+ \left(\frac{-d}{a^2+b^2+c^2}\right) c\right) +d \notag\\
&\quad\notag\\
&= ax+by+cz +\left(\frac{-d}{a^2+b^2+c^2}\right) a^2 +\left(\frac{-d}{a^2+b^2+c^2}\right) b^2 +\left(\frac{-d}{a^2+b^2+c^2}\right) c^2 +d \notag\\
&\quad\notag\\
&= -d + d\notag\\
&\notag\\
&= 0\notag\\
\notag
\end{align}

\newpage

which shows that

\[(\tilde{x},\tilde{y},\tilde{z}) \in v^{\bot} + T\]\

Therefore

\[v^{\bot} + T \subseteq P\]\

In the other direction if $(x,y,z) \in P$ then let

\[ (\tilde{x},\tilde{y},\tilde{z}) = (x,y,z) -T =  (x+ \left(\frac{d}{a^2+b^2+c^2}\right) a, y+ \left(\frac{d}{a^2+b^2+c^2} \right) b, z + \left(\frac{d}{a^2+b^2+c^2}\right) c)\]\

and we have

\begin{align}
<(a,b,c), (\tilde{x},\tilde{y},\tilde{z})> &= (ax+ \left(\frac{d}{a^2+b^2+c^2}\right) a^2, by+ \left(\frac{d}{a^2+b^2+c^2} \right) b^2, cz + \left(\frac{d}{a^2+b^2+c^2}\right) c^2)\notag\\
&\quad\notag\\
&= ax+by + cz +  \left(\frac{d}{a^2+b^2+c^2}\right)(a^2+b^2+c^2)\notag\\
&\quad\notag\\
&= -d +d \notag\\
&\quad\notag\\
&= 0 \notag\\
\notag
\end{align}

so $(\tilde{x},\tilde{y},\tilde{z}) \in v^{\bot}$ and 
 
\[  (x,y,z) = (\tilde{x},\tilde{y},\tilde{z}) + T \in v^{\bot}+T \]\

therefore

\[ P = v^{\bot}+T \]\

Claim 1 is proved.

\newpage

{\large Claim 2: $\quad\quad d(\vec0, P) =  \frac{\vert d\vert}{\sqrt{a^2+b^2+c^2}}$}\\
\\

We have that 

\[ \vec0 \in v^{\bot} \quad \Rightarrow \quad T = \vec0 + T \in v^{\bot} + T\]\

  Furthermore

\begin{align}
\Vert T \Vert_2 &= \sqrt{\left(\frac{-d}{a^2+b^2+c^2} a\right)^2+\left(\frac{-d}{a^2+b^2+c^2} b\right)^2+\left( \frac{-d}{a^2+b^2+c^2} c\right)^2}\notag\\
\quad\notag\\
&= \sqrt{ \left(\frac{-d}{a^2+b^2+c^2}\right)^2(a^2+b^2+c^2) }\notag\\
\quad\notag\\
&= \frac{\vert d\vert}{\sqrt{a^2+b^2+c^2}}\notag\\
\notag
\end{align}

and so 

\[ d(\vec0,P) \geq \frac{\vert d\vert}{\sqrt{a^2+b^2+c^2}}\]\


To obtain the reverse inequality consider 

\[ (\tilde{x},\tilde{y},\tilde{z}) =(x,y,z)  + T \in v^{\bot} + T\]\

then 


\begin{align}
\Vert  (\tilde{x},\tilde{y},\tilde{z}) \Vert_2 &= \sqrt{\left(x-\frac{-d}{a^2+b^2+c^2} a\right)^2+\left(y-\frac{-d}{a^2+b^2+c^2} b\right)^2+\left( z-\frac{-d}{a^2+b^2+c^2} c\right)^2} \notag\\
&\quad\notag\\
&=  \sqrt{x^2+y^2+z^2 -\left(\frac{-2d}{a^2+b^2+c^2}\right)(ax+by+cz)+ \left(\frac{-d}{a^2+b^2+c^2}\right)^2(a^2+b^2+c^2) }\notag\\
&\quad\notag\\
&=  \sqrt{x^2+y^2+z^2 + \left(\frac{-d}{a^2+b^2+c^2}\right)^2(a^2+b^2+c^2) } \quad\quad\quad \text{since}\  (ax+by+cz)=0 \notag\\
\quad\notag\\
&\geq \sqrt{ \left(\frac{-d}{a^2+b^2+c^2}\right)^2(a^2+b^2+c^2)} \quad\quad\quad \text{since}\  x^2+y^2+z^2\geq 0\notag\\
\quad\notag\\
&= \frac{\vert d\vert}{\sqrt{a^2+b^2+c^2}}\notag\\
\notag
\end{align}

so, since $(\tilde{x},\tilde{y},\tilde{z}) $ was arbitrary we have

\[ d(\vec0,P) \geq \frac{\vert d\vert}{\sqrt{a^2+b^2+c^2}} \]\

and we conclude

\[d(\vec0,P) = \frac{\vert d\vert}{\sqrt{a^2+b^2+c^2}}\]\

Claim 2 is proved.


\end{solution}

\newpage







\begin{problem}[Problem 5 Section 1-4  - Do Carmo]
Show that an equation of a plane passing through three noncolinear points 

\[p_1 = (x_1,y_1,z_1) \quad p_2 = (x_2,y_2,z_2) \quad  p_3 = (x_3,y_3,z_3)\]

is given by

\[ (p-p_1)\wedge(p-p_2)\cdot(p-p_3) =0 \]

where $p = (x,y,z)$ is an arbitrary point of the plane and $p-p_1$, for instance, means the vector $(x-x_1,y-y_1,z-z_1)$.
\end{problem}


\begin{solution}
We take as geometrically evident that three non-colinear points determine a unique plane $P$ in $\mathbb{R}^3$.  As discussed in the previous problem there corresponds to $P$ a unique subspace $V_P$ such that

\[ \forall \  p \in P \quad V_P = P - p \]\

or equivalently, since a subspace is closed under scalor multiplication

\[ \forall \  p \in P \quad V_P = p-P \]\

We have that

\[ (p-p_1), (p-p_2), (p-p_3) \in V_p\]\

we consider two cases:\\

Case 1:  $(p-p_1)$ and $(p-p_2)$ are parallel. \\

In this case, the angle $\theta$ between these two vectors is either $0$ or $\pi$ and in either case $sin(\theta)=0$.  So we have

 \[\Vert (p-p_1)\wedge(p-p_2)\Vert = \Vert (p-p_1)\Vert\Vert (p-p_1)\Vert\vert sin(\theta)\vert =0\]\

and it follows that

\[ (p-p_1)\wedge(p-p_2) = \vec0\]

and so

\[ (p-p_1)\wedge(p-p_2)\cdot(p-p_3) = \vec0 \cdot (p- p_3)= 0 \]

as required.\\


Case 2: $(p-p_1)$ and $(p-p_2)$ are not parallel. \\

In this case 

\[ (p-p_1)\wedge(p-p_2) \neq \vec0 \quad \text{and} \quad (p-p_1)\wedge(p-p_2) \in V_P^{\bot}\]\

Notice

\[ 3 = dim(\mathbb{R}^3) = dim(V_P \oplus V_P^{\bot}) = dim(V_P) + dim(V_P^{\bot}) = 2 + dim(V_P^{\bot})\]\

it follows that

\[ dim(V_P^{\bot}) = 1\]\

So, since $(p-p_1)\wedge(p-p_2) $ is a nonzero element of the one dimensional subspace $V_P^{\bot}$ we have

\[ span((p-p_1)\wedge(p-p_2) ) =  V_P^{\bot}\]\

and so

\[ span((p-p_1)\wedge(p-p_2) )^{\bot} =  V_P\]\

so since $(p-p_3) \in V_P=span((p-p_1)\wedge(p-p_2) )^{\bot}$ we have

\[ (p-p_1)\wedge(p-p_2)\cdot(p-p_3) = 0 \]

as required.\\

We have shown

\[ p \in P \quad \Rightarrow \quad  (p-p_1)\wedge(p-p_2)\cdot(p-p_3) = 0 \]\\

\newpage

Next suppose that $p \in \mathbb{R}^3$ satisfies the equation

\[ (p-p_1)\wedge(p-p_2)\cdot(p-p_3) = 0 \]\

then

\begin{equation*}
det
\left[
\begin{matrix}
\leftarrow & (p-p_1) & \rightarrow\\
 \leftarrow & (p-p_2) & \rightarrow\\
\leftarrow &  (p-p_3)& \rightarrow\\
\end{matrix}
\right] 
=0
\end{equation*}\

It follows that the three vectors are linearly dependent.  So 

\[ dim(span\{(p-p_1),(p-p_2),(p-p_3)\}) <3 \]\

Since the vectors $p_1$, $p_2$, and $p_3$ are not colinear, we have that not all three of the vectors $p-p_1$, $p-p_2$, and $p-p_3$ are zero.  Therefore

 \[ dim(span\{(p-p_1),(p-p_2),(p-p_3)\}) \neq 0 \]\

Similarly, since the vectors $p_1$, $p_2$, and $p_3$ are not colinear, we have that

 \[ dim(span\{(p-p_1),(p-p_2),(p-p_3)\}) \neq 1 \]\

Since otherwise there would be a line $L$ which contains all  three of the vectors $p-p_1$, $p-p_2$, and $p-p_3$.  But then the line 
$-L +p$ would contain all three of $p_1$, $p_2$, and $p_3$; a contradiction.\\

We conclude that 

 \[ dim(span\{(p-p_1),(p-p_2),(p-p_3)\}) =2  \]\

It follows that there exists a two dimensional subspace $V$ such that 

\[ p-p_1, p-p_2, p-p_3 \in V\]\

 for which it follows that

\[ p_1-p, \ p_2-p,\  p_3-p \in V\]\

and so 

\[ p_1, \ p_2,\  p_3 \in V +p\]

But $V+p$ is a two dimensional affine subspace and the three noncolinear points $p_1$, $p_2$, and $p_3$ uniquely determine the dimension 2 affine subspace $P$, therefore we have

\[ P = V+p\]\

but $P$ is the translation of a unique two dimensional subspacae $V_P$ defined above, therefore we have

\[ V_P = V\]\

and it follows that 

\[ p-p_1, \ p-p_2, \ p-p_3 \in V_P\]\

and so 

\[ p \in V_P + p_1 = P\]\


We have shown

\[ (p-p_1)\wedge(p-p_2)\cdot(p-p_3) = 0  \quad \Rightarrow \quad  p \in P\]\\

and we may conclude that 

\[ (p-p_1)\wedge(p-p_2)\cdot(p-p_3) = 0  \]\

is an equation for the plane $P$





\end{solution}




\newpage

\begin{problem}[Problem 11(a) Section 1-4  - Do Carmo]
Show that the volume $V$ of a parallelepiped generated by three linearly independent vectors $u,v,w \in \mathbb{R}^3$ is given by 

\begin{equation*}
V = \vert (u\wedge v)\cdot w \vert
\end{equation*}

and use this to introduce an oriented volume in $\mathbb{R}^3$

\end{problem}

\begin{solution}  

We are given three vectors $u$, $v$, and  $w$,  in that order.  I will ignore the assumption that they are linearly independent and only assume that all three of these vectors is nonzero.\\

Consider the vector 
\[u\wedge v = \Vert u \Vert \Vert v \Vert sin(\theta) \vec{n}\]

\begin{itemize}

\item the angle $\theta$ between $v$ and $u$ is choosen so that $0\leq \theta \leq \pi$, and does not reflect the orientation of $v$ and $u$, i.e. wether one moves clockwise or counterclockwise while sweeping from $v$ to $u$. It follows
that $0\leq sin(\theta) \leq 1$.

\item The magnitude $\Vert u \wedge v \Vert = \Vert u \Vert \Vert v \Vert sin(\theta)$ is equal to the area of the parallelogram determined by $u$ and $v$

\item The unit vector $\vec{n}$ is uniquely specified by the requirements that it  be orthogonal to both $u$ and $v$ and  that $u$, $v$, and $n$, in that order, form a right hand system of vectors.

\end{itemize}

Notice that if the pair of vectors $u$ and $v$ form a degenerate parallelogram because they are colinear, then $u \wedge v = \vec0 $.  In this case, for any choice of $w$, we will have
$u\wedge v \cdot w = 0$ which is the correct value for the volume of a degenerate parrallelepiped.\\

Next consider the scalor 
\[ u\wedge v \cdot w = (\Vert u \Vert \Vert v \Vert sin(\theta) \vec{n})\cdot w =  (\Vert u \Vert \Vert v \Vert sin(\theta))\Vert w \Vert cos(\phi) \]

\begin{itemize}

\item the angle $\phi$ between $v\wedge u$ and $w$ is choosen so that $0\leq \theta \leq \pi$, and does not reflect the orientation of $v \wedge u$ and $w$, i.e. it is a positive angle wether one moves clockwise or counterclockwise while sweeping from
${v \wedge u}$ to $w$. It follows that ${-1 \leq cos(\phi) \leq 1}$.

\item The magnitude $\Vert u \wedge v \Vert = \Vert u \Vert \Vert v \Vert sin(\theta)$ is equal to the area of the parallelepiped determined by $v\wedge u$ and $w$

\item The  angel $\phi$ between the vectors $\vec{n}$ and $w$ has positive $cos(\phi)$ when $0\leq \phi < \frac{\pi}{2}$ and is negative $cos(\phi)$ when $\frac{\pi}{2} < \phi \leq \pi$.
so since $\vec{n}$ was choosen to make $u$, $v$, $\vec{n}$ a right hand system. It follows that $u\wedge v \cdot w$ is positive when $u$, $v$, and $w$ form a right hand system and is negative when $u$, $v$, and $w$ form a left hand system.

\end{itemize}

Notice that if the pair of vectors $\vec{n}$ and $w$ have a $90^{\circ}$ angle between them, then $u$, $v$, and $w$ form a degenerate parallelepiped, then $u\wedge v \cdot w = 0$ which is the correct value for the volume of a degenerate parrallelepiped.\\
\newpage


So given three nonzero vectors $u,v,w$ we have that $u\wedge v \cdot w$ does compute a signed volume where

\begin{equation*}
sign(u\wedge v \cdot w) = 
\begin{cases}
1 & u,v,w\ \text{form a right hand system}\\
0 & u,v,w \ \text{form a degenerate parallelepiped}\\
-1 & u,v,w \ \text{form a left hand system}\\
\end{cases}
\end{equation*}\

and, indeed, the volume of the parallelepiped generated by $u,v,w \in \mathbb{R}^3$ is given by

\[ V = \vert u\wedge v \cdot w \vert\]\

as required.



\end{solution}



\begin{problem}[Problem 11(b) Section 1-4  - Do Carmo]
Prove that

\begin{equation*}
V^2 = 
\left\vert
\begin{matrix}
u \cdot u & u\cdot v  & u\cdot w\\
v \cdot u & v \cdot v & v\cdot w\\
w \cdot u & w\cdot v & w\cdot w\\
\end{matrix}
\right\vert
\end{equation*}

\end{problem}
\begin{solution}\
We have that 

\begin{align}
V^2 &= \vert u\wedge v\cdot w \vert^2 \notag\\
\quad\notag\\
&= \left( det(
\left[
\begin{smallmatrix}
\leftarrow& u & \rightarrow \\
\leftarrow & v & \rightarrow \\
\leftarrow & w & \rightarrow \\
\end{smallmatrix}
\right]
)
\right)^2
\notag\\
\quad\notag\\
&=  det(
\left[
\begin{smallmatrix}
\leftarrow& u & \rightarrow \\
\leftarrow & v & \rightarrow \\
\leftarrow & w & \rightarrow \\
\end{smallmatrix}
\right]
)
 det(
\left[
\begin{smallmatrix}
\uparrow& \uparrow & \uparrow \\
u & v & w \\
\downarrow & \downarrow & \downarrow \\
\end{smallmatrix}
\right]
)\notag\\
\quad\notag\\
&=  det(
\left[
\begin{smallmatrix}
\leftarrow& u & \rightarrow \\
\leftarrow & v & \rightarrow \\
\leftarrow & w & \rightarrow \\
\end{smallmatrix}
\right]
\left[
\begin{smallmatrix}
\uparrow& \uparrow & \uparrow \\
u & v & w \\
\downarrow & \downarrow & \downarrow \\
\end{smallmatrix}
\right]
)\notag\\
\quad\notag\\
&= det(
\left[
\begin{matrix}
u \cdot u & u\cdot v  & u\cdot w\\
v \cdot u & v \cdot v & v\cdot w\\
w \cdot u & w\cdot v & w\cdot w\\
\end{matrix}
\right]
)
\notag
\end{align}

as required.
















\end{solution}

\newpage

\begin{problem}[Problem 13 Section 1-4  - Do Carmo)]
Let $u(t) = (u_1(t), u_2(t),u_3(t))$ and $v(t) = (v_1(t),v_2(t),v_3(t))$ be diferentiable maps from the interval $(a,b)$ into $\mathbb{R}^3$. 
If the derivatives $u^{\prime}(t)$ and $v^{\prime}(t)$ satisfy the conditions

\[u^{\prime}(t) = au(t) + bv(t) \quad \text{and} \quad v^{\prime}(t) = cu(t) - a v(t) \]\

where $a$, $b$, and $c$ are constants.  Show that $u(t) \wedge v(t)$ is a constant vector.
	
\end{problem}

\begin{solution}
If we have $u: I_1 \to \mathbb{R}^3$ and $v: I_2 \to \mathbb{R}^3$ then we have that $u\wedge v: I_3 \to \mathbb{R}^3 $ where $I_3 = I_1 \cap I_2$.  So it makes sense to consider the derivative of $u\wedge v$ with respect to time.\\

By the product rule for the wedge product we have

\[ \frac{d}{dt}( u(t)\wedge v(t)) = \frac{du(t)}{dt} \wedge v(t) + u(t)\frac{dv(t)}{dt}\]\

now substituting the above formulas for $ \frac{du(t)}{dt}$ and $\frac{dv(t)}{dt}$ we have

\[ \frac{d}{dt}( u(t)\wedge v(t))  = (au(t) + bv(t))\wedge v(t) + u(t) \wedge (cu(t) - a v(t))\]\

next, using the fact that the wedge product is bilinear, we have

\[\frac{d}{dt}( u(t)\wedge v(t))  = au(t)\wedge v(t)  + bv(t)\wedge v(t) + c u(t) \wedge u(t) - au(t) \wedge v(t))\]\

finally, since $u(t)\wedge u(t) = v(t)\wedge v(t) =0$, we have

\[\frac{d}{dt}( u(t)\wedge v(t)) = \vec0\]\

It follows that  $u(t) \wedge v(t)$ is a constant vector, as required.




\end{solution}

\newpage

\begin{problem}[Problem 1 - Lecture ]
Find the length of the curve obtained by intersecting the sphere $x^2+y^2+z^2 =4$ and the cylinder $(x-1)^2 + y^2 =1$ in 
$\mathbb{R}^3$
\end{problem}

\begin{solution}

\begin{itemize}

\item Parametrize the cylinder and notice that no constraint is placed on $z$.

\[ x(t) = 1+ cos^2t \quad y(t) = sint \quad \text{for} \ t \in \mathbb{R}\]\

\item  When the cylinder is intersected with the sphere a constraint is imposed on $z$.  Find this constraint by substitution of the cylinder constraints on $x$ and $y$ into the equation for the sphere.

\begin{align}
(1+cost)^2 + sin^2t + z^2(t) = 4 &\Rightarrow z^2(t) = 4 - cos^2t - sin^2t - 2cost -1 \notag\\
&\quad\notag\\
&\Rightarrow z^2(t) = 2(1-cost)\notag\\
\notag
\end{align}

\item  We arrive at the following equations for the trace of the  intersection of the cylinder and the sphere.

\[ x(t) = 1+cost \quad y(t) = sint \quad z^2(t) = 2(1-cost)\quad \text{for} \ t \in \mathbb{R}\]\

\item Due to the symmetry of the surfaces being intersected, the trace of the intersection of the sphere and cylinder lies in the four quadrants $Q(x>0, y>0,z>0), Q(x>0,y>0,z<0), Q(x>0,y<0,z<0), Q(x>0,y<0,z>0)$ and
in each of these four quadrants the length of the trace is the same.  In each case, the trace restricted to a quadrant may be obtained fromt he trace in $Q(x>0, y>0,z>0)$ by a combination of rotations and reflections, operations that preserve length. 
Therefore, to compute the length of the trace of the intersection it is enough to find the length of the intersection restricted to $Q(x>0, y>0,z>0)$ and multiply by 4.

\item  Notice that in the quadrant $Q(x>0, y>0,z>0)$  we may replace 

\[z^2(t) = 2(1-cost)\quad \text{with the positive square root} \quad z(t) = \sqrt{2(1-cost)}\]

so we have  the parametrization for this section of the trace 

\[ \alpha(t)= (1+cost, sint, \sqrt{2(1-cost)}) \]\

We need to determine an interval $I$ such that when as $t$ ranges over $I$ the curve $\alpha$ travels over the points in the trace of the intersection exactly once.

\newpage

\item When $0< t < \pi$ we see that $(1+cost) >0, sint>0 $ and by definition $\sqrt{2(1-cost)}>0$.  So with $t$ selected from this range, $\alpha(t)$  travels over the points in the trace of the intersection contained in  $Q(x>0, y>0,z>0)$ exactly once.
So we arrive at a parametrization of the intersection of the sphere and the cylinder restriced to the non-negative orthant $Q(x>0, y>0,z>0)$

\[ \alpha(t) =(1+cost, sint, \sqrt{2(1-cost)})\quad t \in (0,\pi) \]\

\item We have

\[ \alpha^{\prime}(t) = \left(-sint,cost,\frac{sint}{\sqrt{2(1-cost}} \right) \]\

and 
\[ \Vert \alpha^{\prime}(t) \Vert  = \sqrt{ \frac{3+cost}{2}} \]\


\item  So we have the length of the curve obtained by intersecting the sphere $x^2+y^2+z^2 =4$ with the cylinder $(x-1)^2 +y^2 =1$ is given by

\[ 4 \int_0^{\pi} \sqrt{ \frac{3+cost}{2}}  \approx 15.2808 \]

where the approximate value for the integral was provided by multiplying by 4 the result presented in the solutions.  
 \end{itemize}


\end{solution}

\newpage

\begin{problem}[Problem 1 Section 1-2  - Do Carmo)]
Find a parametrized curve $\alpha(t)$ whose trace is the circle $x^2 + y^2 = 1 $ such that $\alpha(t)$ runs clockwise around the circle with $\alpha(0) =(0,1)$
\end{problem}

\begin{solution}

Let $I = \mathbb{R}$ and define $\alpha : I \to \mathbb{R}$  by

\[ \alpha(t) = (sin(t),cos(t)) \]\

Notice

\begin{align}
\alpha(0) &= (sin(0),cos(0)) = (0,1) \notag\\
\alpha(\frac{\pi}{2}) &= (sin(\frac{\pi}{2}),cos(\frac{\pi}{2})) = (1,0) \notag\\
\alpha(\pi) &= (sin(\pi),cos(\pi)) = (0,-1) \notag\\
\alpha(\frac{3\pi}{2}) &= (sin(3\frac{\pi}{2}),cos(3\frac{\pi}{2})) = (-1,0) \notag\\
\alpha(\pi) &= (sin(\pi),cos(\pi)) = (0,-1) \notag\\
\notag
\end{align}

So 

\[\alpha(0) = (sin(0),cos(0)) = (0,1)\]

 and 

\[trace(\alpha) = \alpha(I) = \{(x,y)\in \mathbb{R}^2: x^2 +y^2 =1\}\] \
 
and the circle is transversed clockwise.

\end{solution}



\newpage



\begin{problem}[Problem 3 Section 1-2  - Do Carmo)]
A parametrized curve $\alpha(t)$ has the property that its second derivative $\alpha^{\prime\prime}$ is identically zero.  What can be said about $\alpha$?
\end{problem}

\begin{solution}

We have that 

\[ \alpha^{\prime\prime}(t) = (0 , \dots, 0) \]\

so that

\[ \alpha^{\prime}(t) = ( \int\ dt, \dots, \int\ dt) = (c_1,\dots,c_n)\]\

and

\[ \alpha(t) = (\int c_1\ dt, \dots, \int c_n\ dt) = (c_1t + b_1, \dots, c_nt + b_n)\]\

so we may write 

\[ \alpha(t) = 
\left[
\begin{matrix}
c1\\
\vdots\\
c_n\\
\end{matrix}
\right]
t
+
\left[
\begin{matrix}
b_1\\
\vdots\\
b_n\\
\end{matrix}
\right]
\]\\

which is the equation of a line  through the vector $(b_1,\dots,b_n) \in \mathbb{R}^n$ in the direction of the vector $(c_1,\dots,c_n)\in \mathbb{R}^n$.\\

So we may conclude that if $\alpha^{\prime\prime}(t) = \vec0$ for all $t \in I$, then the trace of $\alpha$ is a subset of a stright line in $\mathbb{R}^n$.


\end{solution}

\newpage

\begin{problem}[Problem 4 Section 1-2  - Do Carmo)]
Let $\alpha: I \to \mathbb{R}$ be a parametrized curve and let $v \in \mathbb{R}^3$ be a fixed vector.  Assume that $\alpha^{\prime}(t)$ is orthogonal to $v$ for all $t \in I$ and that $\alpha(0)$ is also orthogonal to $v$.
Prove that $\alpha(t)$ is orthogonal to $v$ for all $t$.
\end{problem}

\begin{solution}

We have for any $t \in I$ that

\[ <v,\alpha^{\prime}(t)> = v_1\alpha_1^{\prime}(t) + \dots + v_n \alpha_n^{\prime}(t) =0 \]\

so integrating both sides from 0 to $t$ we have

\[ \int_0^t  v_1\alpha_1^{\prime}(u)\ du + \dots + \int_0^t v_n \alpha_n^{\prime}(u)\ du =\int_0^t 0\ du \]\

or evaluating the integrals

\[ v_1 \alpha_1(t) - v_1 \alpha_1(0) + \dots +  v_n \alpha_n(t) - v_n \alpha_n(0)  = 0 \]\

Since  $\alpha_i(0) = 0$ for each $i = 1,\dots, n$, this simplifies to 

\[ v_1 \alpha_1(t) + \dots +  v_n \alpha_n(t)  = 0 \]\

which, written otherwise, is

\[ <v, \alpha(t)> = 0 \]\

so $\alpha(t)$ is orthogonal to $v$ for all $t \in I$.


\end{solution}

\newpage




\begin{problem}[Problem 5 Section 1-2  - Do Carmo)]
Let $\alpha: I \to \mathbb{R}$ be a parametrized curve, with $\alpha^{\prime}(t) \neq 0$ for all $t \in I$.  
Show that $\vert \alpha(t)\vert$ is a nonzero constant if and only if $\alpha(t)$ is orthogonal to $\alpha^{\prime}(t)$ for all $t \in I$.
\end{problem}

\begin{solution}
\quad\\
\quad\\

Two vectors should be orthogonal when the "angle between them" is $90^{\circ}$.   Do Carmo only defines a concept of angle between two vectors $v$ and $w$ when both $v, w \neq \vec0$; and only when this is true do we have the equivalence

\[ v \ \text{and}\ w \ \text{are orthogonal}\quad \iff \quad <v,w>=0\]\\


$\Rightarrow$) $ \quad$ Suppose that for all $t \in I\quad \Vert \alpha(t) \Vert$ is a nonzero constant.

 Notice that for all $t$: $ \quad\Vert \alpha(t)\Vert = C \neq 0 \quad \Rightarrow \quad \alpha(t) \neq \vec0 \quad$.\\

Also we have assumed that  $\alpha^{\prime}(t) \neq 0$.  So, in this case,  we have that

\[ <\alpha(t),  \alpha^{\prime}(t)> = 0\quad  \Rightarrow \quad \alpha(t) \ \text{and} \ \alpha^{\prime}(t) \ \text{are orthogonal}\]\

Therefore it is enough to show that for all $t \in I$

\[ <\alpha(t),  \alpha^{\prime}(t)> = 0\]\

but

\begin{align}
\Vert \alpha(t)\Vert = C \quad &\Rightarrow \quad  <\alpha(t), \alpha(t)> = C^2 \notag\\
&\quad \notag\\
&\Rightarrow\quad  (\alpha_1(t))^2 + \dots + (\alpha_n(t))^2 = C^2 \notag\\
&\quad \notag\\
&\Rightarrow \quad \frac{d}{dt} \left((\alpha_1(t))^2 + \dots + (\alpha_n(t))^2\right) =0 \notag\\
&\quad \notag\\
&\Rightarrow \quad  2\alpha_1(t)\alpha_1^{\prime}(t) + \dots + 2\alpha_n(t)\alpha_n^{\prime}(t)  = 0 \notag\\
&\quad \notag\\
&\Rightarrow \quad  2<\alpha(t),\alpha^{\prime}(t)> = 0 \notag\\
\notag
\end{align}

So for all $t \in I$ we have $\quad <\alpha(t),\alpha^{\prime}(t)> = 0 \quad$ as required.

\newpage

$\Leftarrow$)  $ \quad$   As discussed above, by assuming  that $\alpha$ and $\alpha^{\prime}$ are orthogonal, for all $t \in I$, we have tacitly assumed that 
\[ \alpha(t) \neq \vec0 \quad \text{for all} \ t \in I\]\

Therefore, to show that $\Vert \alpha(t) \Vert$ equals a nonzero constant, it is enough to show that $\Vert \alpha(t) \Vert$  equals a constant.  The fact that such a constant must be nonzero is built into our assumptions.

\begin{align}
\alpha(t)\ \text{and} \ \alpha^{\prime}(t)\ \text{are orthogonal} \quad &\Rightarrow \quad <\alpha(t),\alpha^{\prime}(t)> =0 \notag\\
\quad\notag\\
&\Rightarrow \quad \alpha_1(t)\alpha_1^{\prime}(t) + \dots + \alpha_n(t)\alpha_n^{\prime}(t) \notag = 0\notag\\
\quad\notag\\
&\Rightarrow \int_{t_0}^t\alpha_1(u)\alpha_1^{\prime}(u)\ du+ \dots +\int_{t_0}^t\ \alpha_n(u)\alpha_n^{\prime}(u)\ du \notag = 0\notag\\
\quad\notag\\
&\Rightarrow \sum_{i=1}^n \int_{t_0}^t\alpha_i(u)\alpha_i^{\prime}(u)\ du = 0 \notag\\
\notag
\end{align}

Where $t_0$ is an arbitrary point selected from $I$. Notice that the definite integrals are well defined since both $\alpha_i(t)$ and $\alpha_i^{\prime}$ are continuous, for each $i = 1,\dots,n$.\\
\\
Applying integration by parts to $\ \int_{t_0}^t\alpha_i(u) \alpha_i^{\prime}(u)\ du\ $ with 
$ \quad
\begin{smallmatrix}
u=\alpha_i(t) & dv = \alpha_i^{\prime}(t)\\
du = \alpha_i^{\prime}(t)& \alpha_i(t)\\
\end{smallmatrix}
\quad$
we have \\

\[ \int_{t_0}^t\alpha_i(u)\alpha_i^{\prime}(u)\ du = \left.(\alpha_i(u))^2\right|_{u=t_0}^{u=t} -  \int_{t_0}^t\alpha_i(u)\alpha_i^{\prime}(u)\ du \]

Therefore

\[ \sum_{i=1}^n\left.(\alpha_i(u))^2\right|_{u=t_0}^{u=t} =2\sum_{i=1}^n \int_{t_0}^t\alpha_i(u)\alpha_i^{\prime}(u)\ du \]\

So when $\alpha(t)\ \text{and} \ \alpha^{\prime}(t)$ are orthogonal we have

\begin{align}
\sum_{i=1}^n\left.(\alpha_i(u))^2\right|_{u=t_0}^{u=t} = 0 \quad &\Rightarrow \quad \sum_{i=1}^n(\alpha_i(t))^2 = \sum_{i=1}^n (\alpha_i(t_0))^2 \notag\\
&\quad\notag\\
&\Rightarrow \quad \Vert \alpha(t) \Vert^2 = \Vert \alpha(t_0) \Vert^2 \notag\\
\notag
\end{align}

We conclude that $\Vert \alpha(t) \Vert $ is a constant for all $t \in I$, as required.

\end{solution}











\end{document}

